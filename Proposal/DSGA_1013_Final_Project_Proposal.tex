\documentclass{article}

\usepackage[affil-it]{authblk} 
\usepackage{geometry}
 \geometry{
 a4paper,
 total={160mm,257mm},
 left=20mm,
 top=20mm,
 }

\begin{document}
\title{\large DSGA 1013 Final Project Proposal}
\author{\small Alec Hon, Andrew Yeh}
\date{\footnotesize \today}
\maketitle

\section{Introduction}

Wavelets, or 'small waves', are a family of zero mean functions that are used mostly in signal processing, with all wavelets being characterized by starting with zero amplitude, followed by an increase in amplitude, and then a final decrease in amplitude back to zero over a small time scale. While Fourier Transforms are powerful in representing a function across the entire time domain by transforming a function into trigonometric polynomials, they struggle with studying functions over a local time space as many extra coefficients need to be added into the Fourier Transform of a function to effectively cancel out all the amplitude outside the area of study. On the other hand, Wavelet Analysis expands a function into translations and dilations of a mother wavelet function and allow the study of functions at a local time range,  making it possible to recreate functions with fewer coefficients at a local space than a Fourier transform. Wavelets have been used in image processing, with a large amount of study being focused on image compression and denoising.  
 \newline
 
\noindent In this project, we wish to study the use of wavelet decomposition as a pooling technique in Convolutional Neural Networks (CNNs) as a method for improving image classification in CNNs. Conventional CNN pooling techniques include max pooling and mean pooling, with both pooling techniques showing shortcomings, as details in images are either diluted or lost upon using either pooling algorithm. Other pooling methods include mixed pooling, where max pooling or mean pooling is randomly selected at every convolved field (Williams, Li 2018). Since wavelets are particularly useful in studying signals at a local level, we can consider the convolved field as a local region in an image in which we can use wavelets to decompose each field and return its approximation or detail coefficient as a pooling technique. It has been shown that wavelet decomposition as a pooling technique on convolved layers is an effective technique in image classification, particularly with the Haar Wavelet and the Symlet Wavelet, as shown by Chaabane, Mellouli, Hamdani, et.\ al (2018). As a result, we want to study the usage of different mother wavelets for decomposition as a pooling technique in image classification first for greyscaled images, and then for color images. Our methodology is described below:

\section{Methodology}
1.) Our first goal is to closely recreate the results of Chaabane, Mellouli, Hamdani, et.\ al (2018) in constructing a CNN with wavelet decomposition as a pooling method, utilizing the MNIST dataset as a method to test effective accuracy of wavelet decomposition in image classification. By doing this, we will show that we are able to utilize wavelet decomposition effectively in a CNN. We will compare our accuracies of the wavelet decomposition pooling technique with the standard max, mean, and mixed pooling techniques to see whether or not the wavelet decomposition pooling creates CNNs with higher accuracy. 
\newline

\noindent 2.) We then explore the usage of different wavelet families utilizing the PyWavelet Package for the decomposition step in pooling, seeing which wavelet family allows the CNN to classify more accurately.  Afterwards, time permitting, we will attempt to perform the wavelet decomposition for color images, utilizing the CIFAR-10 dataset for classification. Once again, we will compare our results with standard CNN pooling techniques to see effective performance of our CNN and whether or not it is more accurate than CNNs using conventional pooling techniques.



\end{document}
